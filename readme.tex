\documentclass[]{article}
\usepackage{lmodern}
\usepackage{amssymb,amsmath}
\usepackage{ifxetex,ifluatex}
\usepackage{fixltx2e} % provides \textsubscript
\ifnum 0\ifxetex 1\fi\ifluatex 1\fi=0 % if pdftex
  \usepackage[T1]{fontenc}
  \usepackage[utf8]{inputenc}
\else % if luatex or xelatex
  \ifxetex
    \usepackage{mathspec}
  \else
    \usepackage{fontspec}
  \fi
  \defaultfontfeatures{Ligatures=TeX,Scale=MatchLowercase}
\fi
% use upquote if available, for straight quotes in verbatim environments
\IfFileExists{upquote.sty}{\usepackage{upquote}}{}
% use microtype if available
\IfFileExists{microtype.sty}{%
\usepackage[]{microtype}
\UseMicrotypeSet[protrusion]{basicmath} % disable protrusion for tt fonts
}{}
\PassOptionsToPackage{hyphens}{url} % url is loaded by hyperref
\usepackage[unicode=true]{hyperref}
\hypersetup{
            pdfborder={0 0 0},
            breaklinks=true}
\urlstyle{same}  % don't use monospace font for urls
\usepackage{color}
\usepackage{fancyvrb}
\newcommand{\VerbBar}{|}
\newcommand{\VERB}{\Verb[commandchars=\\\{\}]}
\DefineVerbatimEnvironment{Highlighting}{Verbatim}{commandchars=\\\{\}}
% Add ',fontsize=\small' for more characters per line
\newenvironment{Shaded}{}{}
\newcommand{\KeywordTok}[1]{\textcolor[rgb]{0.00,0.44,0.13}{\textbf{#1}}}
\newcommand{\DataTypeTok}[1]{\textcolor[rgb]{0.56,0.13,0.00}{#1}}
\newcommand{\DecValTok}[1]{\textcolor[rgb]{0.25,0.63,0.44}{#1}}
\newcommand{\BaseNTok}[1]{\textcolor[rgb]{0.25,0.63,0.44}{#1}}
\newcommand{\FloatTok}[1]{\textcolor[rgb]{0.25,0.63,0.44}{#1}}
\newcommand{\ConstantTok}[1]{\textcolor[rgb]{0.53,0.00,0.00}{#1}}
\newcommand{\CharTok}[1]{\textcolor[rgb]{0.25,0.44,0.63}{#1}}
\newcommand{\SpecialCharTok}[1]{\textcolor[rgb]{0.25,0.44,0.63}{#1}}
\newcommand{\StringTok}[1]{\textcolor[rgb]{0.25,0.44,0.63}{#1}}
\newcommand{\VerbatimStringTok}[1]{\textcolor[rgb]{0.25,0.44,0.63}{#1}}
\newcommand{\SpecialStringTok}[1]{\textcolor[rgb]{0.73,0.40,0.53}{#1}}
\newcommand{\ImportTok}[1]{#1}
\newcommand{\CommentTok}[1]{\textcolor[rgb]{0.38,0.63,0.69}{\textit{#1}}}
\newcommand{\DocumentationTok}[1]{\textcolor[rgb]{0.73,0.13,0.13}{\textit{#1}}}
\newcommand{\AnnotationTok}[1]{\textcolor[rgb]{0.38,0.63,0.69}{\textbf{\textit{#1}}}}
\newcommand{\CommentVarTok}[1]{\textcolor[rgb]{0.38,0.63,0.69}{\textbf{\textit{#1}}}}
\newcommand{\OtherTok}[1]{\textcolor[rgb]{0.00,0.44,0.13}{#1}}
\newcommand{\FunctionTok}[1]{\textcolor[rgb]{0.02,0.16,0.49}{#1}}
\newcommand{\VariableTok}[1]{\textcolor[rgb]{0.10,0.09,0.49}{#1}}
\newcommand{\ControlFlowTok}[1]{\textcolor[rgb]{0.00,0.44,0.13}{\textbf{#1}}}
\newcommand{\OperatorTok}[1]{\textcolor[rgb]{0.40,0.40,0.40}{#1}}
\newcommand{\BuiltInTok}[1]{#1}
\newcommand{\ExtensionTok}[1]{#1}
\newcommand{\PreprocessorTok}[1]{\textcolor[rgb]{0.74,0.48,0.00}{#1}}
\newcommand{\AttributeTok}[1]{\textcolor[rgb]{0.49,0.56,0.16}{#1}}
\newcommand{\RegionMarkerTok}[1]{#1}
\newcommand{\InformationTok}[1]{\textcolor[rgb]{0.38,0.63,0.69}{\textbf{\textit{#1}}}}
\newcommand{\WarningTok}[1]{\textcolor[rgb]{0.38,0.63,0.69}{\textbf{\textit{#1}}}}
\newcommand{\AlertTok}[1]{\textcolor[rgb]{1.00,0.00,0.00}{\textbf{#1}}}
\newcommand{\ErrorTok}[1]{\textcolor[rgb]{1.00,0.00,0.00}{\textbf{#1}}}
\newcommand{\NormalTok}[1]{#1}
\IfFileExists{parskip.sty}{%
\usepackage{parskip}
}{% else
\setlength{\parindent}{0pt}
\setlength{\parskip}{6pt plus 2pt minus 1pt}
}
\setlength{\emergencystretch}{3em}  % prevent overfull lines
\providecommand{\tightlist}{%
  \setlength{\itemsep}{0pt}\setlength{\parskip}{0pt}}
\setcounter{secnumdepth}{0}
% Redefines (sub)paragraphs to behave more like sections
\ifx\paragraph\undefined\else
\let\oldparagraph\paragraph
\renewcommand{\paragraph}[1]{\oldparagraph{#1}\mbox{}}
\fi
\ifx\subparagraph\undefined\else
\let\oldsubparagraph\subparagraph
\renewcommand{\subparagraph}[1]{\oldsubparagraph{#1}\mbox{}}
\fi

% set default figure placement to htbp
\makeatletter
\def\fps@figure{htbp}
\makeatother


\date{}

\begin{document}

\section{Android SDK setup}\label{android-sdk-setup}

\subsection{Setup for Linux}\label{setup-for-linux}

\subsubsection{Install Java}\label{install-java}

\texttt{bash\ sudo\ apt-get\ update\ sudo\ dpkg\ -\/-add-architecture\ i386\ sudo\ apt-get\ install\ libbz2-1.0:i386\ sudo\ apt-get\ install\ libc6:i386\ libncurses5:i386\ libstdc++6:i386\ lib32z1\ sudo\ apt-get\ install\ openjdk-8-jdk\ openjdk-8-jre}W

Add JAVA\_HOME to path via \textasciitilde{}/.bashrc

\begin{Shaded}
\begin{Highlighting}[]
\BuiltInTok{export} \VariableTok{JAVA_HOME=}\NormalTok{/usr/lib/jvm/java-8-openjdk-amd64}
\end{Highlighting}
\end{Shaded}

\subsubsection{Install Android Studio}\label{install-android-studio}

Download ZIP archive for Linux from:
https://developer.android.com/studio/install.html 1. move the .zip to
/opt 2. extract it 3. chown the folder to your name 4. chmod 777
studio.sh and run it for the installer

Now the android sdk is installed to \textasciitilde{}/Android/Sdk It's
preferred to add \textasciitilde{}/Android/Sdk folders to your path:

\begin{Shaded}
\begin{Highlighting}[]
\BuiltInTok{export} \VariableTok{PATH=$\{PATH\}}\NormalTok{:~/Android/Sdk/tools}
\BuiltInTok{export} \VariableTok{PATH=$\{PATH\}}\NormalTok{:~/Android/Sdk/platform-tools}
\end{Highlighting}
\end{Shaded}

Run \texttt{android}, install the images (atom, etc) and then navigate
to Tools -\textgreater{} Manage AVDs and create a new image Make sure to
install the android-23 version and confirm it exists in
\textasciitilde{}/Android/Sdk/platforms/

\subsection{Setup for Windows}\label{setup-for-windows}

\subsubsection{Install Java}\label{install-java-1}

Install the most recent
\href{http://www.oracle.com/technetwork/java/javase/downloads/jdk8-downloads-2133151.html}{Java
JDK} (NOT just the JRE).

Next, create an environment variable for JAVA\_HOME pointing to the root
folder where the Java JDK was installed. So, if you installed the JDK
into C:\Program Files\Java\jdk7, set JAVA\_HOME to be this path. After
that, add the JDK's bin directory to the PATH variable as well.
Following the previous assumption, this should be either
\%JAVA\_HOME\%\bin or the full path C:\Program Files\Java\jdk7\bin

\subsubsection{Android SDK}\label{android-sdk}

Install \href{https://developer.android.com/studio/index.html}{Android
Studio}. Detailed installation instructions are on Android's developer
site.

Set the ANDROID\_HOME
(C:\Users\textless{}username\textgreater{}\AppData\Local\Android\Sdk)
environment variable to the location of your Android SDK installation It
is also recommended that you add the Android SDK's tools, tools/bin, and
platform-tools directories to your PATH like this\\
\%ANDROID\_HOME\%\tools \\
\ldots{}

\section{Ionic setup}\label{ionic-setup}

\subsection{Pre-requisties}\label{pre-requisties}

\begin{itemize}
\tightlist
\item
  Node.js v6
\item
  npm v3
\end{itemize}

\section{Install latest cordova and ionic from
npm}\label{install-latest-cordova-and-ionic-from-npm}

\begin{Shaded}
\begin{Highlighting}[]
\ExtensionTok{npm}\NormalTok{ install -g cordova ionic}
\end{Highlighting}
\end{Shaded}

\section{Ionic plugins}\label{ionic-plugins}

\subsection{DB Meter}\label{db-meter}

\begin{verbatim}
This plugin will be used to get decibel values from audio input
\end{verbatim}

\begin{Shaded}
\begin{Highlighting}[]
\NormalTok{$ }\ExtensionTok{ionic}\NormalTok{ cordova plugin add cordova-plugin-dbmeter}
\NormalTok{$ }\ExtensionTok{npm}\NormalTok{ install --save @ionic-native/db-meter}
\end{Highlighting}
\end{Shaded}

\subsection{Geolocation}\label{geolocation}

\begin{verbatim}
This will be used to get location from users devices
\end{verbatim}

\begin{Shaded}
\begin{Highlighting}[]
\NormalTok{$ }\ExtensionTok{ionic}\NormalTok{ cordova plugin add cordova-plugin-geolocation}
\NormalTok{$ }\ExtensionTok{npm}\NormalTok{ install --save @ionic-native/geolocation}
\end{Highlighting}
\end{Shaded}

\section{Running aplication}\label{running-aplication}

First of all please follow the instruction for setup environement on
your pc and only after run the code bellow

\begin{Shaded}
\begin{Highlighting}[]
\ExtensionTok{npm}\NormalTok{ install }
\ExtensionTok{ionic}\NormalTok{ serve}
\end{Highlighting}
\end{Shaded}

\end{document}
